\chapter{むすび} \label{chap:conclusion}
本研究では, 
一般的なサーフェスモデルの3Dモデリングソフトでは面の表と裏する仕様がほとんどである.
現在の実装では, 面を構成する頂点の順番を扱っておらず, 面の法線方向を表現できていないため, 3Dモデリングの同期機構としては不十分である. また, オブジェクトのマテリアル情報やテクスチャ座標情報も扱っていない.
これらを実装することで, クライアントがデータから一意なオブジェクトを描画することができ, 同期編集可能な3Dモデリングを実現できる.

さらに, 3Dモデリングではオブジェクトを移動し, 複数の頂点が一度に更新される場合がある. 本システムでは, データベースのデータの挿入と検索がボトルネックであり, 大量のリクエスト時の同期の成否はサーバのスペックに依存してしまう. CoMayaのようにオブジェクトの状態を操作するAPIを同期する仕組みを併用すると, オブジェクト単位で同期でき, 複数の頂点の更新も可能となると考えられる.
