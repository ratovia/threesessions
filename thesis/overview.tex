\chapter{システム概要}
本システムはオブジェクトが複数の面から構成されるサーフェスモデルを扱う. Differential Synchronizationでは, 同期のため, 接続されたクライアントごとに, クライアントとサーバに, データのシャドウコピーを作る. クライアントとサーバで, 差分の計算とその適用を行い同期を行う.
 \section{データ構造}
 本システムのデータは従来のテキストのように一次元ではなく, 3Dモデリングで用いるオブジェクトや面, 頂点ごとにデータモデルを作成し, そのデータモデル間で依存関係を持つ必要がある.
 また, サーバのデータとそのデータを各クライアントにコピーしたシャドウコピーを区別するために, シーンというデータを定義する.
 シーンのデータはオブジェクトを, オブジェクトは面を, 面は頂点をそれぞれ子にもつ. また, 子のみが親の参照先をもつ.
 このデータ構造によって, 子を削除した場合に親との依存関係も削除できる.親を新しく設定する場合, 子のデータを複製しながらそれぞれに新しく設定する親を参照先に設定する.
 子のデータを複製していくことによって, 関係を削除した際も複製元のデータは残る.
 \section{固有IDの付与}
 各データには, 複数クライアント間でIDの衝突が起こるのを防ぐため, そのデータを作成したクライアントの識別子を組み込んだ固有IDを与える.
 \section{基本命令}
 オブジェクト, 面, 頂点の各データモデルに対して, 作成, 親への参照の追加, 削除の3つの基本命令を実装した.
 これらの命令はシステムに対する最低限の基本命令であり, これらの命令を組み合わせることで, 3Dモデリングで使われる面の分割や押し出しなど, より高度な命令を実現できる.
