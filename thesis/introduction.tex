\chapter{はじめに}
%%%%%%%%%%%%%%%%%%%%%%%%%%%%%%%%%%
%    背景
%%%%%%%%%%%%%%%%%%%%%%%%%%%%%%%%%%
近年, Googleドキュメント\cite{GOOGLEDOCS}やMicrosoft Word Online\cite{WORDONLINE}に代表されるテキスト編集を中心に, 同期編集システムの研究や開発が注目されている.
同期編集では, 同期する際に生じる編集の衝突を解消することが主な課題である.
編集の衝突を解消でき, 広く使われている手法として操作変換がある.
CoMaya\cite{COMAYA}は操作変換を応用して3Dモデリングにおける同期を可能にした.
しかし, CoMayaの操作変換は, 面と頂点, オブジェクトと面などの依存関係がある要素に対する同期は不可能である.
そのためCoMayaではオブジェクト単位の同期に留まっている.
より柔軟に同期編集をするには, 頂点の操作も含めた3次元データを同期する必要がある.
Differential Synchronization\cite{DS}は, 操作変換とは別の同期手法であり, 元々はテキスト編集を対象に開発されたが, 依存関係の扱いが操作変換より容易になると予想される.
%%%%%%%%%%%%%%%%%%%%%%%%%%%%%%%%%%
%    目的
%%%%%%%%%%%%%%%%%%%%%%%%%%%%%%%%%%
\par
本研究では, Differential Synchronizationの同期手法に基づいて, 3次元データの依存関係を考慮した同期編集機構の手法を提案する.
これにより, 複数人で同一の3次元データを同時に編集できる3Dモデリングソフトの開発を可能にする.
%%%%%%%%%%%%%%%%%%%%%%%%%%%%%%%%%%
%    概要
%%%%%%%%%%%%%%%%%%%%%%%%%%%%%%%%%%
\par
本システムはオブジェクトが複数の面から構成されるサーフェスモデルを扱う. Differential Synchronizationでは, 同期のため, 接続されたクライアントごとに, クライアントとサーバに, データのシャドウコピーを作る. クライアントとサーバで, 差分の計算とその適用を行い同期を行う.
%%%%%%%%%%%%%%%%%%%%%%%%%%%%%%%%%%
%    実験と考察
%%%%%%%%%%%%%%%%%%%%%%%%%%%%%%%%%%
\par
本研究はデータを編集するため, クライアントのインターフェイスとして\ref{ope}節の基本命令を実装した.
実験のためにサーバと3つのクライアントを用意し, クライアントごとに任意の基本命令50件をランダムなタイミングで発行した.
 その後, 各クライアントで同期されたデータを比較した結果, データはすべて一致しており, 同期可能であることを確認できた.
%%%%%%%%%%%%%%%%%%%%%%%%%%%%%%%%%%
%    論文の構成
%%%%%%%%%%%%%%%%%%%%%%%%%%%%%%%%%%
\par
本論文の構成は次の通りである. まず, 第2章で本研究の背景と目的について述べる. 次に, 第3章でシステムで利用する手法について述べる. そして, 第4章で実験とその結果について述べ, 考察する. 最後に第5章で本研究のむすびとして今後の課題を述べる.
