\chapter{はじめに}
\lavel{chap:introduction}
%%%%%%%%%%%%%%%%%%%%%%%%%%%%%%%%%%
%    アブストラクト
%%%%%%%%%%%%%%%%%%%%%%%%%%%%%%%%%%
近年, GoogleドキュメントやMicrosoft Word Onlineに代表されるテキスト編集を中心に, 同期編集システムの研究や開発が注目されている.
同期編集では, 同期する際に生じる編集の衝突を解消することが主な課題である.
編集の衝突を解消でき, 広く使われている手法として操作変換がある.
CoMaya\cite{COMAYA}は操作変換を応用して3Dモデリングにおける同期を可能にした.
しかし, \cite{COMAYA}の操作変換は, 面と頂点, オブジェクトと面などの依存関係がある要素に対する同期は不可能である.
そのためCoMayaではオブジェクト単位の同期に留まっている.
より柔軟に同期編集をするには, 頂点の操作も含めた3次元データの同期する必要がある.
Differential Synchronization\cite{DS}は, 操作変換とは別の同期手法であり, 元々はテキスト編集を対象に開発されたが, 依存関係の扱いが操作変換より容易になると予想される.
%%%%%%%%%%%%%%%%%%%%%%%%%%%%%%%%%%
%    目的
%%%%%%%%%%%%%%%%%%%%%%%%%%%%%%%%%%
%%%%%%%%%%%%%%%%%%%%%%%%%%%%%%%%%%
%    概要
%%%%%%%%%%%%%%%%%%%%%%%%%%%%%%%%%%
%%%%%%%%%%%%%%%%%%%%%%%%%%%%%%%%%%
%    実験と考察
%%%%%%%%%%%%%%%%%%%%%%%%%%%%%%%%%%
%%%%%%%%%%%%%%%%%%%%%%%%%%%%%%%%%%
%    論文の構成
%%%%%%%%%%%%%%%%%%%%%%%%%%%%%%%%%%
本論文の構成は次の通りである.
まず, 第
