\documentclass{AIabst}
%\documentclass[master]{AIabst} %修論はこちら
% 論文概要
\input personal
\begin{document}
\makeAbstHeader
%
%
%
\section{背景}
	近年, GoogleドキュメントやMicrosoft Word Onlineに代表されるテキスト編集を中心に, 同期編集システムの研究や開発が注目されている.
	同期編集では, 同期する際に生じる編集の衝突を解消することが主な課題である.
	編集の衝突を解消でき, 広く使われている手法として操作変換がある.
	CoMaya\cite{COMAYA}は操作変換を応用して3Dモデリングにおける同期を可能にした.
	しかし, \cite{COMAYA}の操作変換は, 面と頂点, オブジェクトと面などの依存関係がある要素に対する同期は不可能である.
	そのためCoMayaではオブジェクト単位の同期に留まっている.
	より柔軟に同期編集をするには, 頂点の操作も含めた3次元データの同期する必要がある.
	Differential Synchronization\cite{DS}は, 操作変換とは別の同期手法であり, 元々はテキスト編集を対象に開発されたが, 依存関係の扱いが操作変換より容易になると予想される.
\section{目的}
 本研究では, Differential Synchronizationの同期手法に基づいて, 3次元データの依存関係を考慮した同期編集機構の手法を提案する.
  これにより, 複数人で同一の3次元データを同時に編集できる3Dモデリングソフトの開発を可能にする.
\section{システム}
 本システムはオブジェクトが複数の面から構成されるサーフェスモデルを扱う. Differential Synchronizationでは, 同期のため, 接続されたクライアントごとに, クライアントとサーバに, データのシャドウコピーを作る. クライアントとサーバで, 差分の計算とその適用を行い同期を行う.
  \subsection{データ構造}
	本システムのデータは従来のテキストのように一次元ではなく, 3Dモデリングで用いるオブジェクトや面, 頂点ごとにデータモデルを作成し, そのデータモデル間で依存関係を持つ必要がある.
  また, サーバのデータとそのデータを各クライアントにコピーしたシャドウコピーを区別するために, シーンというデータを定義する.
  シーンのデータはオブジェクトを, オブジェクトは面を, 面は頂点をそれぞれ子にもつ. また, 子のみが親の参照先をもつ.
  このデータ構造によって, 子を削除した場合に親との依存関係も削除できる.親を新しく設定する場合, 子のデータを複製しながらそれぞれに新しく設定する親を参照先に設定する.
  子のデータを複製していくことによって, 関係を削除した際も複製元のデータは残る.
	\subsection{固有IDの付与}
	各データには, 複数クライアント間でIDの衝突が起こるのを防ぐため, そのデータを作成したクライアントの識別子を組み込んだ固有IDを与える.
	\subsection{基本命令}
	オブジェクト, 面, 頂点の各データモデルに対して, 作成, 親への参照の追加, 削除の3つの基本命令を実装した.
	これらの命令はシステムに対する最低限の基本命令であり, これらの命令を組み合わせることで, 3Dモデリングで使われる面の分割や押し出しなど, より高度な命令を実現できる.
\section{実験と考察}
本研究はデータを編集するため, クライアントのインターフェイスとして3.3項の基本命令を実装した.
実験のためにサーバと3つのクライアントを用意し, クライアントごとに任意の基本命令50件をランダムなタイミングで発行した.
 その後, 各クライアントで同期されたデータを比較した結果, データはすべて一致しており, 同期可能であることを確認できた.
\section{むすび}
 現在の実装では, 面を構成する頂点の順番を扱っていないため, 3Dモデリングの同期機構としては不十分である. また, オブジェクトのマテリアル情報も扱っていない.
これらを実装することで, 同期編集可能な3Dモデリングを実現できる.

\begin{thebibliography}{9}
	\bibitem{COMAYA}
	Agustina, et al., ``CoMaya: Incorporating Advanced Collaboration Capabilities into 3D Digital Media Design Tools,'' {\it Proc. CSCW '08}, pp. 5--8, 2008.
	\bibitem{DS}
  Fraser,	``Differential Synchronization,''  {\it Proc. DecEng '09}, pp. 13--20, 2009.
\end{thebibliography}
\end{document}
