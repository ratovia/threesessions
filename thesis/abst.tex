\documentclass{AIabst}
%\documentclass[master]{AIabst} %修論はこちら
% 論文概要
\input personal
\begin{document}
\makeAbstHeader
%
%
%
\section{背景}
	近年, 同期編集についてGoogleドキュメントやMicrosoft Word Onlineに代表されるテキスト編集に関するシステムや3Dモデリングを対象にした同期編集の研究が注目されている.
	同期編集の研究において, 同期する際に生じる編集の衝突を解消することは主な課題となっている.
	編集の衝突を解消でき, 広く使われている手法として操作変換がある.
	CoMaya\cite{COMAYA}は操作変換を応用して3Dモデリングにおける同期を可能にした.
	しかし, 既存の操作変換は, 面と頂点, オブジェクトと面などの依存関係がある要素に対する同期は不可能である.
	そのためCoMayaではオブジェクト単位の同期となる.
	より柔軟に同期編集をするには, 頂点の操作も含めた3次元データを同期する必要がある.
	そこで, 操作変換以外にDifferentional Syncronization(DS)\cite{DS}といった同期手法が研究されている.
	しかし, テキスト編集における同期手法のため, 3次元データに適用するためには依存関係を考慮する必要がある.

\section{目的}
 本研究では, DSの同期手法に基づいて, 3次元データの依存関係を考慮した同期編集機構の手法を提案する.
  これにより, 複数人で3次元データを共有し, 同期編集可能な3Dモデリングソフトの開発を目指す.
\section{システム}
 本システムは同期手法としてDSを採用し, 編集したデータを管理するサーバプログラムと, 内部データの差を計算し送信するクライアントによって構成される.
  また, サーバの内部データを従来のテキストと同様に一次元の空間によって管理すると依存関係があった場合にデータの管理が困難となる.
  3Dモデリングで用いるエンティティごとにデータモデルを作成し, そのデータモデル間で依存関係を持つことによって解決する.
  \subsection{データ構造}
  内部データと各クライアントのシャドウコピーを区別するために, シーンというデータモデルを定義する.
  また, 3Dモデリングにおけるオブジェクトを表現するためメッシュ, 面, 頂点, というデータモデルを用意する.
  シーンのデータモデルはメッシュ, 面, 頂点を子にもち, メッシュは面を, 面は頂点を子にもつ. また, 子のみが親の参照先をもつ. このデータ構造によって, 子を削除した場合に親との依存関係も削除できる.親を新しく設定する場合, 子のデータを複製しながらそれぞれに新しく設定する親を参照先を設定する. 子のデータを複製していくことによって, 関係を削除した際もデータは残る.
	\subsection{固有IDの付与}
	データの検索を固有IDを各データに振ることで可能にした. 同期の際に複数クライアント間で衝突が起こるのを防ぐためにクライアントの識別子を固有IDに組み込む.
	\subsection{基本命令}
	メッシュ, 面, 頂点の各データモデルに作成, 親への参照の追加, 削除の三つの命令を定義した. これらの命令をシステムに対する最低限の基本命令とし, ユーザインターフェイスを作る際はこれらの命令を組み合わせることで, 3Dモデリングで使われる面の分割や押し出しなど, より高度な命令を実現できる.
\section{実験と考察}
 本研究は内部データを編集するため, クライアントのインターフェイスとして3.3の基本命令を実装した.
 3つのクライアントごとに任意の命令50件をランダムなタイミングで発行した.
 その後, 各クライアントに同期された内部データを比較し一致しているか確かめた結果, 内部データが一致し同期可能であることを確認できた.
\section{むすび}
 現在の問題点としてオブジェクトのマテリアル情報がない点や, 面を構成する頂点の順番がない点があげられ, 3Dモデリングの同期機構としては不十分である.
これらを実装することで, 3Dモデリングをより柔軟に同期ができると考えられる.

\begin{thebibliography}{9}
	\bibitem{COMAYA}
	Agustina, et al., ``CoMaya: Incorporating Advanced Collaboration Capabilities into 3D Digital Media Design Tools,'' {\it CSCW '08}, pp. 5--8, 2008.
	\bibitem{DS}
  Fraser, Neil,	``Differential Synchronization''  {\it Proc.ACM DecEng '09}, pp. 13--20, 2009.
\end{thebibliography}
\end{document}
