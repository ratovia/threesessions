\chapter{実験と考察} \label{chap:test}
%%%%%%%%%%%%%%%%%%%%%%%%%%%%%%%%%%
%    実験の目的
%%%%%%%%%%%%%%%%%%%%%%%%%%%%%%%%%%
\section{実験の目的}
Differential Syncronizationに基づき, 本研究での提案手法による同期機構で3次元データは同期可能であるかどうか確かめる.
%%%%%%%%%%%%%%%%%%%%%%%%%%%%%%%%%%
%    実験環境
%%%%%%%%%%%%%%%%%%%%%%%%%%%%%%%%%%
\section{実験の準備}
データを編集するためにクライアントを用意し, クライアントのインターフェイスとして\ref{ope}節の基本命令と, \ref{固有id}節の固有IDを付与する機能を実装した.
%%%%%%%%%%%%%%%%%%%%%%%%%%%%%%%%%%
%    実験環境
%%%%%%%%%%%%%%%%%%%%%%%%%%%%%%%%%%
\section{実験環境}
本実験で使用したサーバの計算機環境を表\ref{server}に示す. また3つのクライアントの計算機環境を表\ref{client1}, 表\ref{client2}, 表\ref{client3}に示す.
% ネットワーク
\begin{table}[htbp]
\begin{center}
	\caption{使用するサーバのスペック}
	\begin{tabular}{|l|l|} \hline
		OS & aaa \\ \hline
		CPU & aaa \\ \hline
		メモリ & aaa \\ \hline
    開発言語 & aaa \\ \hline
		データベース & \\ \hline
	\end{tabular}
	\label{server}
\end{center}
\end{table}

\begin{table}[htbp]
\begin{center}
	\caption{使用するクライアント1のスペック}
	\begin{tabular}{|l|l|} \hline
		OS & aaa \\ \hline
		CPU & aaa \\ \hline
		メモリ & aaa \\ \hline
    開発言語 & aaa \\ \hline
	\end{tabular}
	\label{client1}
\end{center}
\end{table}

\begin{table}[htbp]
\begin{center}
	\caption{使用するクライアント2のスペック}
	\begin{tabular}{|l|l|} \hline
		OS & aaa \\ \hline
		CPU & aaa \\ \hline
		メモリ & aaa \\ \hline
    開発言語 & aaa \\ \hline
	\end{tabular}
	\label{client2}
\end{center}
\end{table}

\begin{table}[htbp]
\begin{center}
	\caption{使用するクライアント3のスペック}
	\begin{tabular}{|l|l|} \hline
		OS & aaa \\ \hline
		CPU & aaa \\ \hline
		メモリ & aaa \\ \hline
    開発言語 & aaa \\ \hline
	\end{tabular}
	\label{client3}
\end{center}
\end{table}

%%%%%%%%%%%%%%%%%%%%%%%%%%%%%%%%%%
%    実験方法
%%%%%%%%%%%%%%%%%%%%%%%%%%%%%%%%%%
\section{実験方法}
クライアントを3つ用意し, 各クライアントごとに任意の基本命令50件を5分以内のランダムなタイミングで発行した. また, クライアントは4秒ごとに差分を計算しサーバからの返信が着き次第適用処理を行う. 最後のクライアントが実験を開始して5分10秒後に各クライアントのデータが一致しているか確かめる.
%%%%%%%%%%%%%%%%%%%%%%%%%%%%%%%%%%
%    実験結果
%%%%%%%%%%%%%%%%%%%%%%%%%%%%%%%%%%
\section{実験結果}

%%%%%%%%%%%%%%%%%%%%%%%%%%%%%%%%%%
%    考察
%%%%%%%%%%%%%%%%%%%%%%%%%%%%%%%%%%
\section{考察}
