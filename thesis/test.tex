\chapter{実験と考察} \label{chap:test}
%%%%%%%%%%%%%%%%%%%%%%%%%%%%%%%%%%
%    実験の目的
%%%%%%%%%%%%%%%%%%%%%%%%%%%%%%%%%%
\section{実験の目的}
Differential Syncronizationに基づき, 本研究での提案手法による同期機構で3次元データは同期可能であるかどうか確かめるため実験を行う.
%%%%%%%%%%%%%%%%%%%%%%%%%%%%%%%%%%
%    実験環境
%%%%%%%%%%%%%%%%%%%%%%%%%%%%%%%%%%
\section{実験環境}
データを編集するため, クライアントのインターフェイスとして\ref{ope}節の基本命令と, サーバと同様に\ref{固有id}節の固有IDを付与した.
本実験で使用したサーバの計算機環境を表\ref{server}に示す. また3つのクライアントの計算機環境を表\ref{client1}, 表\ref{client2}, 表\ref{client3}に示す.
%%%%%%%%%%%%%%%%%%%%%%%%%%%%%%%%%%
%    実験方法
%%%%%%%%%%%%%%%%%%%%%%%%%%%%%%%%%%
\section{実験方法}
実験のためにサーバと3つのクライアントを用意し, クライアントごとに任意の基本命令50件をランダムなタイミングで発行した.

%%%%%%%%%%%%%%%%%%%%%%%%%%%%%%%%%%
%    実験結果
%%%%%%%%%%%%%%%%%%%%%%%%%%%%%%%%%%
\section{実験結果}
%%%%%%%%%%%%%%%%%%%%%%%%%%%%%%%%%%
%    考察
%%%%%%%%%%%%%%%%%%%%%%%%%%%%%%%%%%
\section{考察}
