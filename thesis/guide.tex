\documentclass[11pt]{jarticle}
%\documentclass[11pt]{jreport}
\usepackage{AIthesis}
\begin{document}
\pagestyle{empty}
\begin{center}
{\Large\bf
%平成 25 年度 
修士論文・卒業論文の書式について}
\end{center}
%\begin{flushright}
%\begin{minipage}{60mm}
%知能情報工学科\\
%学務委員補佐\\
%片峯 恵一\\
%情報科学専攻知能分野\\
%大学院委員補佐\\
%榎田 修一
%\end{minipage}
%\end{flushright}
%
\section{論文の書式}
\noindent
論文は, A4 版で, コピー用紙程度の上質紙に印字すること. 
書式は, 過去の修士論文, 卒業論文等を参考にし, 指導教員の指示を仰ぐこと. 
% この文書自体も通常の論文の形式({\tt bachelor.sty\/})に従っている.
%
\subsection{使用言語}
\noindent
論文を記述するのに使用する言語は, 日本語または英語とする. 
%
\subsection{ページのレイアウト}
\noindent
製本その他読みやすさ等を考慮して, マージンは大きめにとること.

\vspace*{3mm}
\hspace*{10mm}
\begin{tabular}{ll}
上マージン & 25mm 程度 \\
下マージン & 30mm 程度 (ページ番号もマージン内に含む) \\
左マージン & 35mm 程度 (製本の都合上 30mm 以上は必要) \\
右マージン & 25mm 程度 \\
\end{tabular}

\subsection{文字の大きさ}
\noindent
読みやすさ等を考慮して, 極端に小さい文字や大きな文字はさけ, 
行間は十分にあけること. 
文字サイズ 11〜12pt, 1ページ 30 行で
日本語の場合は 1 行あたり 40 文字程度が目安となる. 

\subsection{製本方法}
\noindent
提出する論文は学科事務室で配布するバインダを用いて製本する. 
(バインダは, 修士論文の場合は提出前に事前に,
卒業論文の場合は卒業論文概要提出の際に受け取ること.)
ただし, 綴じ穴は製本の都合上 2 穴, 穴の位置は紙の端から 12mm とする. 

バインダーの表紙および背に, 
{\tt cover1.tex\/}および{\tt spine.tex\/}
を用いて作成した表紙, 背表紙を貼り付けること. 
また, 論文の最初のページに{\tt cover2.tex\/}で作成した
中表紙(指導教員の押印が必要), 
次のページに{\tt abst.tex\/}で作成した概要
(この綴じ込む分については指導教員の押印は不要)を綴じ, 
それ以降に論文本体を綴じること. 

\subsection{提出について}
\noindent
提出についての詳細は
修士論文は {\tt Mschedule.pdf},
卒業論文は {\tt Bschedule.pdf} を参照のこと.

\end{document}

